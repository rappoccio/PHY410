\documentclass{article}
\usepackage{amsmath}
\usepackage{enumitem}
\usepackage[utf8]{inputenc}
\begin{document}
Herbert Ludowieg\\
PHY 410
\section{Problem 2}
\begin{enumerate}[label=(\alph*)]
\item 1010
\item Max even numbers gotten by binary representation 256 and 512 with 9 and 10 bits respectively.\\
    Therefore the value for 436 is between 1 0000 0000 and 10 0000 0000. Must be 9 bits long.\\
    \begin{equation*}
    \begin{aligned}[c]
    2^8 &= 256 \\
    2^7 &= 128 \\
    \end{aligned}
    \qquad
    \begin{aligned}[c]
    2^6 &= 64  \\
    2^5 &= 32  \\
    \end{aligned}
    \qquad
    \begin{aligned}[c]
    2^4 &= 16  \\
    2^3 &= 8   \\
    \end{aligned}
    \qquad
    \begin{aligned}[c]
    2^2 &= 4   \\
    2^1 &= 2   \\
    \end{aligned}
    \end{equation*}
    \begin{equation*}
    1*2^8+1*2^7+0*2^6+1*2^5+1*2^4+0*2^3+1*2^2+0*2^1 = 436
    \end{equation*}
    Then the bit representation is 1 1011 0100
\item 1024 can be represented as a power of two to be exact it is 2 to the 10.\\
    So the bit representation is 100 0000 0000.
\item
    \begin{equation*}
    -13_{10} = -1101_{2}
    \end{equation*}
    Flip the bits,\\
    \begin{equation*}
    0010_{2}
    \end{equation*}
    Add one to the bit reqresentation,\\
    \begin{equation*}
    0011_{2}
    \end{equation*}
\item
    \begin{align*}
    -1023_{10} &= -11 1111 1111_{2}\\
               &= 00 0000 0000_{2} \\
               &= 00 0000 0001_{2} \\
    \end{align*}
\item
    \begin{align*}
    -1024_{10} &= -100 0000 0000_{2}\\
               &= 011 1111 1111_{2} \\
               &= 100 0000 0000_{2} \\
    \end{align*}
\end{enumerate}
\end{document}
